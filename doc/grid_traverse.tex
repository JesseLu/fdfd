\documentclass{article}
\usepackage{amsmath}
\usepackage{hyperref}

\title{Traversing a three-dimensional grid with PyCuda}
\author{Jesse Lu}
\begin{document}
\maketitle


\section{Problem statement}
We wish to efficiently traverse a three-dimensional grid of size $(xx,yy,zz)$
    in order to perform an operation at every point.
However the efficiency of such a traversal is dependent upon many variables
    which often cannot be known or chosen,
    such as memory layout and chip architecture.
Therefore, our strategy is to simply try multiple traversal configurations,
    and choose the fastest one.

We choose to use PyCuda\footnote{\url{http://mathema.tician.de/software/pycuda}}
    for this task, in order to take advantage of both the parrallelism of CUDA,
    and the metaprogramming capabilities achievable in Python.


\section{Traversal options}
\begin{description}
\item[shape $(s_x, s_y, s_z)$] determines the size of the three-dimensional box
    of grid points that a single multiprocessor operates on at once.
\item[cardinality $c = (m_1, m_2, m_3)$] determines the order of directions 
    in which the update proceeds. 
    Specifically, the the $m_1$ direction is updated first,
        and the $m_3$ direction is updated last.
\end{description}


\section{How it works}


\end{document}
